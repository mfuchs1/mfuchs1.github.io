% Created 2020-04-27 Mo 22:25
% Intended LaTeX compiler: pdflatex
\documentclass[pdftex,a4paper,12pt,bibliography=totoc,draft]{scrartcl}
\author{Matthias Fuchs}
\date{\today}
\title{}
\hypersetup{
 pdfauthor={Matthias Fuchs},
 pdftitle={},
 pdfkeywords={},
 pdfsubject={},
 pdfcreator={Emacs 26.3 (Org mode 9.3.6)}, 
 pdflang={English}}
\begin{document}

\tableofcontents

\section{02 post}
\label{sec:org1d437fd}
Hier werde ich meinen Text weiterschreiben.

\begin{align}
  G &= m_{S} g\\
  [m_{S}] &= 1 kg\\
  G &= 1 kg \cdot 9,81 \frac{m}{sec^2} = 9,81 N = 1 kp
\end{align}


\section{Nikola Workflow}
\label{sec:org81795e8}

\begin{enumerate}
\item Im Ordner \emph{home/matthias/blog/mfuchs1.github.io} ein Terminal öffnen.
\item git checkout sources
\item Diesen Befehl eingeben: nikola new\textsubscript{post} -f orgmode <dateiname>.org
\item Titel eingeben
\item Emacs starten
\item Den neuen "Eintrag" (ein org.file) öffnen
\item Text schreiben\ldots{}
\item Zurück zum Terminal
\item Folgende git-Befehle eingeben:
\item git add ./posts/<dateiname>.org
\item git commit -m "hier kommt etwas hinein"
\item git push --all
\item git push -u origin master
\item nikola build
\item nikola serve
\item nikola github\textsubscript{deploy}
\end{enumerate}

Die Blog-Seite ist auf folgender URL: \url{https://mfuchs1.github.io/}

\section{Zum Schluss}
\label{sec:org821980d}
noch ein schönes Foto:

\begin{figure}[htbp]
\centering
\includegraphics[width=.9\linewidth]{../../images/blacklion.jpg}
\caption{\label{fig:org69650e5}black lion}
\end{figure}
\end{document}